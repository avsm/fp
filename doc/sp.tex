\documentclass{article}
\usepackage{xspace}
\begin{document}
\title{A Specification of the StructParsing Language}
\author{Prashanth Mundkur}
\date{\today} \maketitle

\newcommand{\DDC}{$\mathrm{DDC}^\alpha$\xspace}
\newcommand{\OCaml}{\textrm{OCaml}\xspace}

\section{Introduction}

The Meta Packet Language (MPL) was presented in
\cite{Madhavapeddy:thesis} to provide concise declarative
specifications of binary packet formats, from which \OCaml code can be
generated for parsers for, and producers of, binary data satisfying
those formats.  The MPL language is simple but expressive, and quite
effective in achieving its task.  Its implemention is fast and has
been used for an impressive set of applications.  However, the
language was presented descriptively, without an underlying semantics,
leaving some details and quirks of the language lacking a rationale,
and its implementation code without a roadmap.

The primary objective for this document is to rectify this situation,
and provide a semi-formal specification for a language StructParsing
(SP) closely derived from the Meta Packet Language
(MPL)~\cite{Madhavapeddy:thesis}, one that can provide both some
rationale for its design and also a guide and some documentation for
its implementation.  The techniques used are inspired by the Data
Description Calculus \DDC presented in \cite{Mandelbaum:thesis}.
Specifically, the dependent-sums of \DDC will be used as the core of
the semantics of SP.

The semantics of \DDC are based on the use of types to describe data.
Each type constructed in the \DDC calculus has three roles: (i) to
describe a collection of byte strings, (ii) to specify datatypes in
the host language (i.e. the target language for the generated code)
corresponding to the internal representation in the host language for
these byte strings, and (iii) to specify the code transforming the
byte strings into their internal representation as values of the
corresponding datatypes.  This will inform our view of specifications
in SP as a collection of types in a dependent type theory, and the
semantics will directly specify the \OCaml types used for the internal
representation, as well as the parsing code transforming the
corresponding byte strings into values of these types.

In addition, MPL specifications are also used
in~\cite{Madhavapeddy:thesis} to specify the efficient construction of
the byte string from an \OCaml value.  Hence, the types in the
specification of SP will also directly specify this reverse
transformation.

\section{Language Specification}

% Use the syntax spec generated by Ott here.

\subsection{Type checking}

\section{Extensions}

\subsection{Alignment}
\subsection{Variants}
\subsection{Classifications}

\section{Packet Construction}

\section{Language Interface}

\bibliographystyle{unsrt}
\bibliography{bib}
\end{document}
